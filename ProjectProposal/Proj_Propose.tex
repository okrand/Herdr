\documentclass{article}
%Packages
\usepackage[margin=1in]{geometry}
\usepackage{setspace}
\usepackage[leftmargin = 1in, rightmargin = 0in, vskip = 0in]{quoting}
\usepackage{microtype}

\begin{document}
\setstretch{2}
\raggedright

%Essay Heading
Orkun Krand \\ Dr. Shubham Jain \\ CS541 - App Development for Smart Devices \\ January 31, 2018
    
    
%Title
	\centerline{Herdr App} 

Herdr app brings people together NOW. Apps like Meetup bring groups of people together at a planned time and place. But there is no solution to "I am bored right now". It will be similar to Meetup in that people can meet for various reasons whether it's playing sports or getting drinks. But Meetup requires you to set up a time a place, invite people and so on. I want to simulate inviting someone out to play a game of racketball. Whether it's someone you know or not. Consider the following example:
\hfill \linebreak

It's Saturday afternoon. It's beautiful out and you want to play badminton. You have a place to play but none of your friends know how to play or are as good as you. So like any millennial in your position, you go online to find someone to play badminton with. Well, you can post on Facebook but you'll likely get responses from your buddy who lives 3 states over or your aunt who just wants to say hi. You try meetup.com but those are all for group events and they aren't happening right now. You Google: ``Want to play badminton right now'' and find absolutely nothing useful. So you give up and spend the day in your house, staring at the tv.
\hfill \linebreak

This is where Herdr comes into play. Herdr allows you to go on your phone and find people who want to do the same activity as you do whether it is playing badminton, trying a new restaurant, or playing a jam session with a band. Herdr is simple too. It uses Facebook to log in so you don't have to create a long profile with details. Once logged in, you pick a category and you can look for activities in a certain radius (entered by you) of where you are (using GPS). In the case of badminton, you would pick ``Sports'' category and type ``badminton'' into the search bar. This shows you a list of people who also want to play badminton, how far they are, how good they are at playing, and how long their post has been up.
\hfill \linebreak 

Creating a new activity is very simple. The user selects the category of the new activity, selects where the activity will take place on a map and adds some details about the activity such when they want to perform the activity and how many people they are looking for. Then, the activity is posted and visible to users searching for the same category of activity in a certain radius. The owner of the activity is taken to the chat room in case they want to type some information about it for people who will join.
\hfill \linebreak

Each activity comes with its chat room. The owner of the event is also the admin of that event's chat room. They can lock/unlock it or kick people out of it as they please. Profiles are public so everyone can see each other's profile and references. After each event, participants are encouraged to write references for one another. This helps create a safe environment for future activities.
\hfill \linebreak

The initial categories are sports, and entertainment. These are bound to change as the application evolves. Within each category, search boxes help users find the specific activity they want. Each activity post shows when it was posted, how many people are needed, when the activity will take place and a short description of the activity. It also shows how many people are already going to the activity.
\hfill \linebreak

I expect 80\% of users will see value in Herdr to become regular users. I am targeting Android API 19+ which covers 90.1\% of users. As my target audience are millennials or people in their 20-30s, this should be reasonable. It does feature GPS location so ideally, the user will allow me to locate them. It also expects users to have Facebook accounts to eliminate the profile builder page and making the app simpler. Plus, I have enough profiles online as do most people my age. 
\hfill \linebreak

You can think of Herdr as a mashup of Meetup, Tinder, and the hangout feature of Couchsurfing. The idea is the same as Meetup, interface will be similar to Tinder, and the NOW feature is where Couchsurfing comes into play. I believe Herdr will bring people together and help us spend our times doing what we like, meeting new people, and being social.

\end{document}